%  sample eprint article in LaTeX           --- M. Peskin, 9/7/00


\documentclass[12pt]{article}
\usepackage{graphicx}
\usepackage{amssymb}


%%%%%%%%%%%%%%%%%%%%%%%%%%%%%%%%%%%%%%%%%%%%%%%%%%%%%%%%%%%%%%%%%%%%
% basic data for the eprint:
%%%%%%%%%%%%%%%%%%%%%%%%%%%%%%%%%%%%%%%%%%%%%%%%%%%%%%%%%%%%%%%%%%%%

\textwidth=6.0in  \textheight=8.25in

%%  Adjust these for your printer:
\leftmargin=-0.3in   \topmargin=-0.20in

%% preprint number data:
\newcommand\pubnumber{SNSN-323-63}
\newcommand\pubdate{\today}

%%  address and funding acknowledgement data:
\def\institute{on behalf of the ATLAS and CMS Collaborations,\\Department of Physics and Astronomy\\ University of Manchester}

%%%%%%%%%%%%%%%%%%%%%%%%%%%%%%%%%%%%%%%%%%%%%%%%%%%%%%%%%%%%%%%%%%%%%%%%%%%%
%   document style macros
%%%%%%%%%%%%%%%%%%%%%%%%%%%%%%%%%%%%%%%%%%%%%%%%%%%%%%%%%%%%%%%%%%%%%%%%%%%%
\def\Title#1{\begin{center} {\Large #1 } \end{center}}
\def\Author#1{\begin{center}{ \sc #1} \end{center}}
\def\Address#1{\begin{center}{ \it #1} \end{center}}
\def\andauth{\begin{center}{and} \end{center}}
\def\submit#1{\begin{center}Submitted to {\sl #1} \end{center}}
\newcommand\pubblock{\rightline{\begin{tabular}{l} \pubnumber\\
         \pubdate  \end{tabular}}}
\newenvironment{Abstract}{\begin{quotation}  }{\end{quotation}}
\newenvironment{Presented}{\begin{quotation} \begin{center} 
             PRESENTED AT\end{center}\bigskip 
      \begin{center}\begin{large}}{\end{large}\end{center} \end{quotation}}
\def\Acknowledgements{\bigskip  \bigskip \begin{center} \begin{large}
             \bf ACKNOWLEDGEMENTS \end{large}\end{center}}
%%%%%%%%%%%%%%%%%%%%%%%%%%%%%%%%%%%%%%%%%%%%%%%%%%%%%%%%%%%%%%%%%%%%%%%%%%%%
%  personal abbreviations and macros
%    the following package contains macros used in this document:
\input econfmacros.tex
%%%%%%%%%%%%%%%%%%%%%%%%%%%%%%%%%%%%%%%%%%%%%%%%%%%%%%%%%%%%%%%%%%%%%%%%%%%

\begin{document}
\begin{titlepage}
\pubblock

\vfill
\Title{Bottlenecks in precision top-quark measurements}
\vfill
\Author{ Valentina Vecchio}
\Address{\institute}
\vfill
\begin{Abstract}
%ATLAS title
%"Treatment of systematic uncertainties in precision top-quark measurements (including correlations in differential cross-sections among different distributions)”

%CMS abstract
%Using the large data sample of Run-II LHC, the treatment of systematic uncertainties plays a crucial role in the precision era of the top quark measurements. Uncertainties can be determined in situ, profiled in the fit, externalised and/or determined from pseudo-experiments. A selected set of precise top quark measurements (cross section?, direct top mass?, top width?) from ATLAS and CMS is presented where justifications are provided and discussed for every treatment.

A review of some of the recent \emph{top}-quark precision measurements at $\sqrt{s}$ = 13 TeV performed by the ATLAS and CMS Collaborations is presented. The different approaches for the determination of the systematic uncertainties are discussed. 


\end{Abstract}
\vfill
\begin{Presented}
$13^\mathrm{th}$ International Workshop on Top Quark Physics\\
Durham, UK (videoconference), 14--18 September, 2020
\end{Presented}
\vfill
\end{titlepage}
\def\thefootnote{\fnsymbol{footnote}}
\setcounter{footnote}{0}
%

\section{Introduction}
The \emph{top}-quark was observed for the first time in proton-antiproton ($p\bar{p}$) collisions at the Tevatron by the CDF and D0 Collaborations 
\cite{Abe_1995,Abachi_1995}. It is the heaviest Standard Model (SM) particle known: its average mass, computed by combining the measurements performed by the Tevatron and the Large Hadron Collider (LHC) experiments \cite{atlas2014combination}, is 
\begin{equation}
m_{\mathrm{top}} = 173.34 \pm 0.27(\mathrm{stat}) \pm 0.71(\mathrm{syst})~\mathrm{GeV}.
\end{equation}
Due to its mass the \emph{top}-quark has the largest Yukawa coupling and its lifetime is smaller than the hadronisation time. From this latter fact, it derives that the \emph{top}-quark is never confined in bound states (\emph{hadrons}), offering the unique opportunity to study the properties of a ``free" quark.
For these reasons, the \emph{top}-quark is considered to be a good probe to test the SM and beyond and, since the first observation, its properties have been studied in great detail, both at the Tevatron and the LHC.  
At the LHC, \emph{top}-quarks are mainly produced in pairs of top-antitop ($t\bar{t}$) with a cross-section of around 800 $pb$ at $\sqrt{s}$=13 TeV.
The large amount of \emph{top}-quarks produced at the LHC opened a new era of precision measurements, where systematic uncertainties dominate the final results. Systematic uncertainties arise in all physics analyses where there exist uncertainties in the predictions of any of the Monte Carlo simulation steps. Uncertainties arising from the detector simulation are normally referred to as \emph{experimental}, whereas any uncertainty that arises in any of the other steps is usually classified as \emph{theoretical}. In the measurements reviewed in this work the main contribution to the systematic uncertainty is given by the theoretical component.
The big impact of the theoretical uncertainties prompted the reaction of both the theoretical and experimental comunities which in the last years released higher older calculations and made use of data to tune the simulations.
Moreover, different techniques for the estimation of theoretical uncertainties have been developed. In this work, a set of precise \emph{top}-quark measurements performed by the ATLAS and CMS experiments is presented, where the different approaches for the systematic uncertainties estimation are compared and discussed.
Two measurements of the inclusive cross-section of $t\bar{t}$ events which make use of a \emph{Profile Likelihood Fit} are discussed. Here, informations on the impact of systematic uncertainties can be extracted from data, the so-called \emph{profiling}, and different techniques are proposed. Morevover, several examples of differential cross-section measurementes are showed, where the derivation of the Covariance Matrix implies or not the usage of toy experiments. Finally, two examples of unfolding measurements in which systematic uncertainties are profiled are discussed. 

\section{Systematic uncertainties in Profile Likelihood Fits}
In this section two measurements are discussed where a Profile Likelihood (PL) Fit is used.
A Likelihood Fit is a statistical inference techniques where a prediction model of the variables under study is constructed. The probability of the observed data under this prediction model is what it is what is usually referred to as \emph{Likelihood}.
A simple example is a counting experiment where the signal strenght $\mu$ of a process is extracted from the distribution of observed data. In this case, the signal (S) and background (B) events are assumed to follow a Poissonian distribution: the Likelihood of the observed data in a given bin \emph{i}, $n_i$, is
\begin{equation}
\mathcal{L}(n_i|\mu) = \mathcal{P}(n_i|\mu\cdot S_i+B_i)
\end{equation}
In this case, the signal strenght is the physics parameter of interest (POI) of the model. Eventually, other parameters can be added to a Likelihood Fit to account for any effect that might change the signal and/or background predictions. 
These parameters are usually referred to as \emph{nuisance parameters} and often account for the effect of systematic uncertainties. When systematic uncertainties are added in the statistical model in the form of nuisance parameters $\theta$ they are said to be \emph{profiled}. The idea behind is that systematic uncertainties on the measurement of a POI come from imperfect knowledge of parameters of the model. It is a common practice to included systematic uncertainties in PL Fits as constrained nuisance parameters, where a some a priori knowledge is implied interpreted as "auxiliary/subsidiary measurement", implemented as constraint/penalty term. The penalty term is in the form of a probability density function: usually it is a Gaussian distribution, however also log-normal, gamma or uniform distributions can be used. 
In the case of a Gaussian constraint, the form of the likelihood becomes 
\begin{equation}
\mathcal{L}(\mu,\vec{\theta}) = \displaystyle\prod_{i=1}^{\mathrm{bins}}\mathcal{P}(n_i|\mu\cdot S_i(\vec{\theta})+B_i(\vec{\theta}))\cdot\displaystyle\prod_{t=1}^{\mathrm{syst}}\mathcal{G}(\theta_t)
\end{equation}

 This statistical framework is largely used to estimate various \emph{top}-quark properties such as the inclusive cross-section \cite{Aad_2020,CMS_inclusive}, the width \cite{ATLAS:2019onj} of the mass \cite{CMS_inclusive}.  
This work will focus on the the measurement of the inclusive $t\bar{t}$ production cross-sections performed in the \emph{lepton+jets} final state by the ATLAS Collaboration \cite{Aad_2020} and in the \emph{dileptonic} final state by the CMS Collaboration \cite{CMS_inclusive}. In both these measurements the inclusive cross-section is extracted by means of a PL fit: however, different strategies are implied on the estimation of the systematic uncertainties. 
In these analyses the inclusive cross-section is extracted from the data in a fiducial phase-space. The fiducial phase space is defined by the analysis requirements on the objects and the events. The cross-section estimated in the fiducial phace-space is proportinal to the inclusive cross-section
\begin{equation}
\sigma_{fid} = A_{fid} \cdot \sigma_{incl}, 
\end{equation}
where the acceptance $A_{fid}$ is defined as the ratio of $t\bar{t}$ events passing the fiducial region cuts over the total number of $t\bar{t}$ events.


\section{Unfolding: definition of Covariance Matrix}

\section{Combining unfolding and profiling}

\section{Conclusions}

%%%%%%%%%%%%%%%%%%%%%%%%%%%%%%%%%%%%%%%%%%%%%%%%%%%%%%%%%%%%%%%%%%%%%%%%%
%%
%%   use this format to include an .eps figure into your paper
%%
%\begin{figure}[htb]
%\centering
%%\includegraphics[height=1.5in]{magnet}
%\caption{Plan of the magnet used in the mesmeric studies.}
%\label{fig:magnet}
%\end{figure}
%%%%%%%%%%%%%%%%%%%%%%%%%%%%%%%%%%%%%%%%%%%%%%%%%%%%%%%%%%%%%%%%%%%%%%%%%%%




%%%%%%%%%%%%%%%%%%%%%%%%%%%%%%%%%%%%%%%%%%%%%%%%%%%%%%%%%%%%%%%%%%%%%%%%%
%%
%%   use this format to include a LaTeX table  into your paper
%%
%\begin{table}[t]
%\begin{center}
%\begin{tabular}{l|ccc}  
%Patient &  Initial level($\mu$g/cc) &  w. Magnet &  
%w. Magnet and Sound \\ \hline
% Ferrando di N. &  0.15     &     0.11      &  $< 0.0005$ \\ \hline
%\end{tabular}
%\caption{Blood cyanide levels for the two patients.}
%\label{tab:blood}
%\end{center}
%\end{table}
%%%%%%%%%%%%%%%%%%%%%%%%%%%%%%%%%%%%%%%%%%%%%%%%%%%%%%%%%%%%%%%%%%%%%%%%%%%




\begin{thebibliography}{99}

%%
%%  bibliographic items can be constructed using the LaTeX format in SPIRES:
%%    see    http://www.slac.stanford.edu/spires/hep/latex.html
%%  SPIRES will also supply the CITATION line information; please include it.
%%


\bibitem{Abe_1995}
CDF Collaboration,``Observation of top quark production in $p\bar{p}$ collisions with the collider detector at Fermilab", {\em Physical Review Letters}, vol.~74, p.~2626–2631, Apr 1995.

\bibitem{Abachi_1995}
D0 Collaboration,``Observation of the top quark", {\em Physical Review Letters}, vol.~74, p.~2632–2637, Apr 1995.

\bibitem{atlas2014combination}
T.~ATLAS, CDF, CMS, and D.~Collaborations,``First combination of Tevatron and LHC measurements of the top-quark mass", 2014.

\bibitem{Aad_2020}
ATLAS~Collaboration, ``Measurement of the $t\bar{t}$ production cross-section in the lepton+jets channel at $\sqrt{s}$=13 tev with the atlas experiment,''
  {\em Physics Letters B}, vol.~810, p.~135797, Nov 2020.

\bibitem{CMS_inclusive}
CMS~Collaboration, ``Measurement of the $t\bar{t}$ production cross section, the top quark mass, and the strong coupling constant using dilepton events in pp collisions
  at $\sqrt{s}$ = 13 TeV", {\em The European
  Physical Journal C}, vol.~79, no.~5, p.~368, 2019.

\bibitem{kCranmer}
Kyle Cranmer, Practical Statistics for the LHC, 2015

\bibitem{ATLAS:2019onj}
``{Measurement of the top-quark decay width in top-quark pair events in the
  dilepton channel at $\sqrt{s}=13$ TeV with the ATLAS detector},'' 8 2019.


\end{thebibliography}

 
\end{document}

