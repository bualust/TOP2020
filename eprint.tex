%  sample eprint article in LaTeX           --- M. Peskin, 9/7/00


\documentclass[12pt]{article}
\usepackage{graphicx}

%%%%%%%%%%%%%%%%%%%%%%%%%%%%%%%%%%%%%%%%%%%%%%%%%%%%%%%%%%%%%%%%%%%%
% basic data for the eprint:
%%%%%%%%%%%%%%%%%%%%%%%%%%%%%%%%%%%%%%%%%%%%%%%%%%%%%%%%%%%%%%%%%%%%

\textwidth=6.0in  \textheight=8.25in

%%  Adjust these for your printer:
\leftmargin=-0.3in   \topmargin=-0.20in

%% preprint number data:
\newcommand\pubnumber{SNSN-323-63}
\newcommand\pubdate{\today}

%%  address and funding acknowledgement data:
\def\institute{on behalf of the ATLAS and CMS Collaborations,\\Department of Physics and Astronomy\\ University of Manchester}

%%%%%%%%%%%%%%%%%%%%%%%%%%%%%%%%%%%%%%%%%%%%%%%%%%%%%%%%%%%%%%%%%%%%%%%%%%%%
%   document style macros
%%%%%%%%%%%%%%%%%%%%%%%%%%%%%%%%%%%%%%%%%%%%%%%%%%%%%%%%%%%%%%%%%%%%%%%%%%%%
\def\Title#1{\begin{center} {\Large #1 } \end{center}}
\def\Author#1{\begin{center}{ \sc #1} \end{center}}
\def\Address#1{\begin{center}{ \it #1} \end{center}}
\def\andauth{\begin{center}{and} \end{center}}
\def\submit#1{\begin{center}Submitted to {\sl #1} \end{center}}
\newcommand\pubblock{\rightline{\begin{tabular}{l} \pubnumber\\
         \pubdate  \end{tabular}}}
\newenvironment{Abstract}{\begin{quotation}  }{\end{quotation}}
\newenvironment{Presented}{\begin{quotation} \begin{center} 
             PRESENTED AT\end{center}\bigskip 
      \begin{center}\begin{large}}{\end{large}\end{center} \end{quotation}}
\def\Acknowledgements{\bigskip  \bigskip \begin{center} \begin{large}
             \bf ACKNOWLEDGEMENTS \end{large}\end{center}}
%%%%%%%%%%%%%%%%%%%%%%%%%%%%%%%%%%%%%%%%%%%%%%%%%%%%%%%%%%%%%%%%%%%%%%%%%%%%
%  personal abbreviations and macros
%    the following package contains macros used in this document:
\input econfmacros.tex
%%%%%%%%%%%%%%%%%%%%%%%%%%%%%%%%%%%%%%%%%%%%%%%%%%%%%%%%%%%%%%%%%%%%%%%%%%%

\begin{document}
\begin{titlepage}
\pubblock

\vfill
\Title{Bottlenecks in precision top-quark measurements}
\vfill
\Author{ Valentina Vecchio}
\Address{\institute}
\vfill
\begin{Abstract}
%ATLAS title
%"Treatment of systematic uncertainties in precision top-quark measurements (including correlations in differential cross-sections among different distributions)”

%CMS abstract
%Using the large data sample of Run-II LHC, the treatment of systematic uncertainties plays a crucial role in the precision era of the top quark measurements. Uncertainties can be determined in situ, profiled in the fit, externalised and/or determined from pseudo-experiments. A selected set of precise top quark measurements (cross section?, direct top mass?, top width?) from ATLAS and CMS is presented where justifications are provided and discussed for every treatment.

A review of some of the recent \emph{top}-quark precision measurements at $\sqrt{s}$ = 13 TeV performed by the ATLAS and CMS Collaborations is presented. The different approaches for the determination of the systematic uncertainties are discussed. 


\end{Abstract}
\vfill
\begin{Presented}
$13^\mathrm{th}$ International Workshop on Top Quark Physics\\
Durham, UK (videoconference), 14--18 September, 2020
\end{Presented}
\vfill
\end{titlepage}
\def\thefootnote{\fnsymbol{footnote}}
\setcounter{footnote}{0}
%

\section{Introduction}
The \emph{top}-quark was observed for the first time in proton-antiproton ($p\bar{p}$) collisions at the Tevatron by the CDF and D0 Collaborations 
\cite{Abe_1995,Abachi_1995}. Since its first observation, this particle has been studied in great detail, both at the Tevatron and the the Large Hadron Collider (LHC). The \emph{top}-quark is the heaviest elementary particle of the Standard Model (SM): the average mass, computed by combining the measurements performed by the Tevatron and the LHC experiments \cite{atlas2014combination}, is 
\begin{equation}
m_{\mathrm{top}} = 173.34 \pm 0.27(\mathrm{stat}) \pm 0.71(\mathrm{syst})~\mathrm{GeV}.
\end{equation}
Due to its large mass the \emph{top}-quark has the largest Yukawa coupling with the Higgs boson and its lifetime is smaller than the hadronisation time. It derives that, contrary to what happens to the lighter quarks, the \emph{top}-quark is never confined in bound states (\emph{hadrons}). This peculiarity offers the unique opportunity to study the properties of a ``free" quark.
The \emph{top}-quark is considered a good probe to both test the SM and to search for hints of physics beyond the SM. For this reason its properties are extensively measured by the LHC experiments.\\ 
At the LHC, \emph{top}-quarks are mainly produced in pairs of top-antitop ($t\bar{t}$) with a cross-section of around 800$pb$ at $\sqrt{s}$=13 TeV.
The large amount of \emph{top}-quarks produced at the LHC opened to a new era of precision measurements, where systematic uncertainties dominate the final results. Systematic uncertainties arise in all physics analyses where there exist uncertainties in the predictions of any of the Monte Carlo simulation steps. Uncertainties arising from the detector simulation are normally referred to as \emph{experimental}, whereas any uncertainty that arises in any of the other steps is usually classified as \emph{theoretical}. In most of the \emph{top}-quark related measurements, the main contribution to the systematic uncertainty is given by the theoretical uncertainties.\\
For this reason, in the past years there was a big effort in both the theoretical and experimental communities to improve the theoretical: higher older calculations have been realised and the data have been used to tune the simulations.
Moreover, different techniques for the estimation of theoretical uncertainties have been developed. In this work, a set of precise \emph{top}-quark measurements performed by the ATLAS and CMS experiments is presented, where the different approaches for the systematic uncertainties estimation are compared and discussed.\\
The measurements of the inclusive and differential cross-sections of $t\bar{t}$ events 

\section{Treatment of systematic uncertainties in statistical analyses}




%%%%%%%%%%%%%%%%%%%%%%%%%%%%%%%%%%%%%%%%%%%%%%%%%%%%%%%%%%%%%%%%%%%%%%%%%
%%
%%   use this format to include an .eps figure into your paper
%%
%\begin{figure}[htb]
%\centering
%%\includegraphics[height=1.5in]{magnet}
%\caption{Plan of the magnet used in the mesmeric studies.}
%\label{fig:magnet}
%\end{figure}
%%%%%%%%%%%%%%%%%%%%%%%%%%%%%%%%%%%%%%%%%%%%%%%%%%%%%%%%%%%%%%%%%%%%%%%%%%%




%%%%%%%%%%%%%%%%%%%%%%%%%%%%%%%%%%%%%%%%%%%%%%%%%%%%%%%%%%%%%%%%%%%%%%%%%
%%
%%   use this format to include a LaTeX table  into your paper
%%
%\begin{table}[t]
%\begin{center}
%\begin{tabular}{l|ccc}  
%Patient &  Initial level($\mu$g/cc) &  w. Magnet &  
%w. Magnet and Sound \\ \hline
% Ferrando di N. &  0.15     &     0.11      &  $< 0.0005$ \\ \hline
%\end{tabular}
%\caption{Blood cyanide levels for the two patients.}
%\label{tab:blood}
%\end{center}
%\end{table}
%%%%%%%%%%%%%%%%%%%%%%%%%%%%%%%%%%%%%%%%%%%%%%%%%%%%%%%%%%%%%%%%%%%%%%%%%%%




\begin{thebibliography}{99}

%%
%%  bibliographic items can be constructed using the LaTeX format in SPIRES:
%%    see    http://www.slac.stanford.edu/spires/hep/latex.html
%%  SPIRES will also supply the CITATION line information; please include it.
%%


\bibitem{Abe_1995}
CDF Collaboration,``Observation of top quark production in $p\bar{p}$ collisions with the collider detector at Fermilab", {\em Physical Review Letters}, vol.~74, p.~2626–2631, Apr 1995.

\bibitem{Abachi_1995}
D0 Collaboration,``Observation of the top quark", {\em Physical Review Letters}, vol.~74, p.~2632–2637, Apr 1995.

\bibitem{atlas2014combination}
T.~ATLAS, CDF, CMS, and D.~Collaborations,``First combination of Tevatron and LHC measurements of the top-quark mass", 2014.


\end{thebibliography}

 
\end{document}

