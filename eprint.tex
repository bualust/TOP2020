%  sample eprint article in LaTeX           --- M. Peskin, 9/7/00


\documentclass[12pt]{article}
\usepackage{graphicx}

%%%%%%%%%%%%%%%%%%%%%%%%%%%%%%%%%%%%%%%%%%%%%%%%%%%%%%%%%%%%%%%%%%%%
% basic data for the eprint:
%%%%%%%%%%%%%%%%%%%%%%%%%%%%%%%%%%%%%%%%%%%%%%%%%%%%%%%%%%%%%%%%%%%%

\textwidth=6.0in  \textheight=8.25in

%%  Adjust these for your printer:
\leftmargin=-0.3in   \topmargin=-0.20in

%% preprint number data:
\newcommand\pubnumber{SNSN-323-63}
\newcommand\pubdate{\today}

%%  address and funding acknowledgement data:
\def\institute{on behalf of the ATLAS and CMS Collaborations,\\Department of Physics and Astronomy\\ University of Manchester}

%%%%%%%%%%%%%%%%%%%%%%%%%%%%%%%%%%%%%%%%%%%%%%%%%%%%%%%%%%%%%%%%%%%%%%%%%%%%
%   document style macros
%%%%%%%%%%%%%%%%%%%%%%%%%%%%%%%%%%%%%%%%%%%%%%%%%%%%%%%%%%%%%%%%%%%%%%%%%%%%
\def\Title#1{\begin{center} {\Large #1 } \end{center}}
\def\Author#1{\begin{center}{ \sc #1} \end{center}}
\def\Address#1{\begin{center}{ \it #1} \end{center}}
\def\andauth{\begin{center}{and} \end{center}}
\def\submit#1{\begin{center}Submitted to {\sl #1} \end{center}}
\newcommand\pubblock{\rightline{\begin{tabular}{l} \pubnumber\\
         \pubdate  \end{tabular}}}
\newenvironment{Abstract}{\begin{quotation}  }{\end{quotation}}
\newenvironment{Presented}{\begin{quotation} \begin{center} 
             PRESENTED AT\end{center}\bigskip 
      \begin{center}\begin{large}}{\end{large}\end{center} \end{quotation}}
\def\Acknowledgements{\bigskip  \bigskip \begin{center} \begin{large}
             \bf ACKNOWLEDGEMENTS \end{large}\end{center}}
%%%%%%%%%%%%%%%%%%%%%%%%%%%%%%%%%%%%%%%%%%%%%%%%%%%%%%%%%%%%%%%%%%%%%%%%%%%%
%  personal abbreviations and macros
%    the following package contains macros used in this document:
\input econfmacros.tex
%%%%%%%%%%%%%%%%%%%%%%%%%%%%%%%%%%%%%%%%%%%%%%%%%%%%%%%%%%%%%%%%%%%%%%%%%%%

\begin{document}
\begin{titlepage}
\pubblock

\vfill
\Title{Bottlenecks in precision top-quark measurements}
\vfill
\Author{ Valentina Vecchio}
\Address{\institute}
\vfill
\begin{Abstract}
%ATLAS title
%"Treatment of systematic uncertainties in precision top-quark measurements (including correlations in differential cross-sections among different distributions)”

%CMS abstract
%Using the large data sample of Run-II LHC, the treatment of systematic uncertainties plays a crucial role in the precision era of the top quark measurements. Uncertainties can be determined in situ, profiled in the fit, externalised and/or determined from pseudo-experiments. A selected set of precise top quark measurements (cross section?, direct top mass?, top width?) from ATLAS and CMS is presented where justifications are provided and discussed for every treatment.

A review of some of the recent top-quark precision measurements at $\sqrt{s}$ = 13 TeV performed by the ATLAS and CMS Collaborations is presented. The different approaches for the determination of the systematic uncertainties are discussed. 


\end{Abstract}
\vfill
\begin{Presented}
$13^\mathrm{th}$ International Workshop on Top Quark Physics\\
Durham, UK (videoconference), 14--18 September, 2020
\end{Presented}
\vfill
\end{titlepage}
\def\thefootnote{\fnsymbol{footnote}}
\setcounter{footnote}{0}
%

\section{Introduction}
The top-quark was observed for the first time in proton-antiproton ($p\bar{p}$) collisions at the Tevatron by the D0 and CDF Collaborations %
\cite{Abazov_2009,Aaltonen_2009}. 
The mass of the top-quark is 
The top-quark mass is the heaviest elementary particle in the Standard Model: due to its mass
The top-quark, which was first observed in proton-antiproton collisions at the Tevatron, is the heaviest elementary particle 
The large data sample delivered by the Large Hadron Collider during its Run 2 opened to a new era of top-quark precision measurements. The treatment of the systematic uncertainties plays a crucial role in the determination of the cross-section and fundamental properties of the top-quark. A set of precise top-quark measurements performed by the ATLAS and CMS experiments is presented, where the different techinques for the systematic uncertainties estimation are discussed.


%%%%%%%%%%%%%%%%%%%%%%%%%%%%%%%%%%%%%%%%%%%%%%%%%%%%%%%%%%%%%%%%%%%%%%%%%
%%
%%   use this format to include an .eps figure into your paper
%%
%\begin{figure}[htb]
%\centering
%%\includegraphics[height=1.5in]{magnet}
%\caption{Plan of the magnet used in the mesmeric studies.}
%\label{fig:magnet}
%\end{figure}
%%%%%%%%%%%%%%%%%%%%%%%%%%%%%%%%%%%%%%%%%%%%%%%%%%%%%%%%%%%%%%%%%%%%%%%%%%%




%%%%%%%%%%%%%%%%%%%%%%%%%%%%%%%%%%%%%%%%%%%%%%%%%%%%%%%%%%%%%%%%%%%%%%%%%
%%
%%   use this format to include a LaTeX table  into your paper
%%
%\begin{table}[t]
%\begin{center}
%\begin{tabular}{l|ccc}  
%Patient &  Initial level($\mu$g/cc) &  w. Magnet &  
%w. Magnet and Sound \\ \hline
% Ferrando di N. &  0.15     &     0.11      &  $< 0.0005$ \\ \hline
%\end{tabular}
%\caption{Blood cyanide levels for the two patients.}
%\label{tab:blood}
%\end{center}
%\end{table}
%%%%%%%%%%%%%%%%%%%%%%%%%%%%%%%%%%%%%%%%%%%%%%%%%%%%%%%%%%%%%%%%%%%%%%%%%%%




\begin{thebibliography}{99}

%%
%%  bibliographic items can be constructed using the LaTeX format in SPIRES:
%%    see    http://www.slac.stanford.edu/spires/hep/latex.html
%%  SPIRES will also supply the CITATION line information; please include it.
%%


\bibitem{Abazov_2009}
D0 Collaboration, ``Observation of
  single top-quark production,'' {\em Physical Review Letters}, vol.~103, Aug
  2009.

\bibitem{Aaltonen_2009}
CDF Collaboration,
  ``Observation of electroweak single top-quark production,'' {\em Physical
  Review Letters}, vol.~103, Aug 2009.



\end{thebibliography}

 
\end{document}

