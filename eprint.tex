%  sample eprint article in LaTeX           --- M. Peskin, 9/7/00


\documentclass[12pt]{article}
\usepackage{graphicx}

%%%%%%%%%%%%%%%%%%%%%%%%%%%%%%%%%%%%%%%%%%%%%%%%%%%%%%%%%%%%%%%%%%%%
% basic data for the eprint:
%%%%%%%%%%%%%%%%%%%%%%%%%%%%%%%%%%%%%%%%%%%%%%%%%%%%%%%%%%%%%%%%%%%%

\textwidth=6.0in  \textheight=8.25in

%%  Adjust these for your printer:
\leftmargin=-0.3in   \topmargin=-0.20in

%% preprint number data:
\newcommand\pubnumber{SNSN-323-63}
\newcommand\pubdate{\today}

%%  address and funding acknowledgement data:
\def\institute{Department of Physics and Astronomy\\ University of Manchester}

%%%%%%%%%%%%%%%%%%%%%%%%%%%%%%%%%%%%%%%%%%%%%%%%%%%%%%%%%%%%%%%%%%%%%%%%%%%%
%   document style macros
%%%%%%%%%%%%%%%%%%%%%%%%%%%%%%%%%%%%%%%%%%%%%%%%%%%%%%%%%%%%%%%%%%%%%%%%%%%%
\def\Title#1{\begin{center} {\Large #1 } \end{center}}
\def\Author#1{\begin{center}{ \sc #1} \end{center}}
\def\Address#1{\begin{center}{ \it #1} \end{center}}
\def\andauth{\begin{center}{and} \end{center}}
\def\submit#1{\begin{center}Submitted to {\sl #1} \end{center}}
\newcommand\pubblock{\rightline{\begin{tabular}{l} \pubnumber\\
         \pubdate  \end{tabular}}}
\newenvironment{Abstract}{\begin{quotation}  }{\end{quotation}}
\newenvironment{Presented}{\begin{quotation} \begin{center} 
             PRESENTED AT\end{center}\bigskip 
      \begin{center}\begin{large}}{\end{large}\end{center} \end{quotation}}
\def\Acknowledgements{\bigskip  \bigskip \begin{center} \begin{large}
             \bf ACKNOWLEDGEMENTS \end{large}\end{center}}
%%%%%%%%%%%%%%%%%%%%%%%%%%%%%%%%%%%%%%%%%%%%%%%%%%%%%%%%%%%%%%%%%%%%%%%%%%%%
%  personal abbreviations and macros
%    the following package contains macros used in this document:
\input econfmacros.tex
%%%%%%%%%%%%%%%%%%%%%%%%%%%%%%%%%%%%%%%%%%%%%%%%%%%%%%%%%%%%%%%%%%%%%%%%%%%

\begin{document}
\begin{titlepage}
\pubblock

\vfill
\Title{Bottlenecks in precision top-quark measurements}
\vfill
\Author{ Valentina Vecchio}
\Address{\institute}
\vfill
\begin{Abstract}
%"Treatment of systematic uncertainties in precision top-quark measurements (including correlations in differential cross-sections among different distributions)”

%Using the large data sample of Run-II LHC, the treatment of systematic uncertainties plays a crucial role in the precision era of the top quark measurements. Uncertainties can be determined in situ, profiled in the fit, externalised and/or determined from pseudo-experiments. A selected set of precise top quark measurements (cross section?, direct top mass?, top width?) from ATLAS and CMS is presented where justifications are provided and discussed for every treatment.

The large data sample delivered by the Large Hadron Collider during its Run 2 opened to a new era of top-quark precision measurements. The treatment of the systematic uncertainties plays a crucial role in the determination of the cross-section and fundamental properties of the top-quark. A set of precise top quark measurements performed by the ATLAS and CMS experiments is presented, where the determination of systematic uncertainties is discussed.

\end{Abstract}
\vfill
\begin{Presented}
$13^\mathrm{th}$ International Workshop on Top Quark Physics\\
Durham, UK (videoconference), 14--18 September, 2020
\end{Presented}
\vfill
\end{titlepage}
\def\thefootnote{\fnsymbol{footnote}}
\setcounter{footnote}{0}
%

\section{Introduction}

Recently, F.  A. Mesmer \index{Mesmer}
has reported evidence of a profound influence of 
magnets on a variety of aspects of human and animal physiology~\cite{Mesmer}.
Numerous authors have claimed to reproduce certain of the phenomena 
reported by Mesmer~\cite{diCenzo,Muller}, but there is little understanding
of the precise mode of action of the magnetic influence.  In this
article, I will present evidence that mesmeric influence can disrupt the
lethal action of sodium cyanide (NaCN) when modulated by a sequence of 
high-pitched tones.  After a description of the phenomenon, I will comment
on the implications for the nature of the magnetic coupling.

\section{Observations}

My observations began when I was called on an emergency basis to aid two
Albanian noblemen, Guglielmo B. and Ferrando di N., who had
ingested the poison in an apparent attempt to commit suicide.  (The facts
of the case were rather mysterious, since these young men had been seen in the 
harbor earlier in the day, affecting a carefree attitude.  However, this 
need not concern us here.)  The standard remedy would have been to induce
vomiting; however, this would have offended the 
sensibilities \index{sensibilities}  of two 
noble ladies present, the daughters of the duke in whose household this 
unfortunate incident took place.  Thus, I thought it wise to begin by 
administering magneto-therapy using the devices that I had at hand.

%%%%%%%%%%%%%%%%%%%%%%%%%%%%%%%%%%%%%%%%%%%%%%%%%%%%%%%%%%%%%%%%%%%%%%%%%
%%
%%   use this format to include an .eps figure into your paper
%%
\begin{figure}[htb]
\centering
\includegraphics[height=1.5in]{magnet}
\caption{Plan of the magnet used in the mesmeric studies.}
\label{fig:magnet}
\end{figure}
%%%%%%%%%%%%%%%%%%%%%%%%%%%%%%%%%%%%%%%%%%%%%%%%%%%%%%%%%%%%%%%%%%%%%%%%%%%

The source of magnetic influence used in this therapy was the simple
iron pole magnet shown in Figure~\ref{fig:magnet}.  The device, roughly
the size of human palm, was held three fingers' length away from each
of the patients, while the blood cyanide level was monitored.  As this 
operator proceeded in silence, only a very small effect on the poison was
observed.  In a moment, however, the sisters began to weep and ululate
in the most painful manner.  Under this blanket of sound, the effects of 
the cyanide reaction were observed to be completely reversed.  Indeed,
even before the end of a single session of treatment, the Albanians were
seen to wake from their comas and offer thanks for their rescue.  Some
sample blood cyanide levels are reported in Table~\ref{tab:blood}.



%%%%%%%%%%%%%%%%%%%%%%%%%%%%%%%%%%%%%%%%%%%%%%%%%%%%%%%%%%%%%%%%%%%%%%%%%
%%
%%   use this format to include a LaTeX table  into your paper
%%
\begin{table}[t]
\begin{center}
\begin{tabular}{l|ccc}  
Patient &  Initial level($\mu$g/cc) &  w. Magnet &  
w. Magnet and Sound \\ \hline
 Guglielmo B.  &   0.12     &     0.10      &     0.001  \\
 Ferrando di N. &  0.15     &     0.11      &  $< 0.0005$ \\ \hline
\end{tabular}
\caption{Blood cyanide levels for the two patients.}
\label{tab:blood}
\end{center}
\end{table}
%%%%%%%%%%%%%%%%%%%%%%%%%%%%%%%%%%%%%%%%%%%%%%%%%%%%%%%%%%%%%%%%%%%%%%%%%%%

\section{Interpretation}

A description of these observations, obviously based on second-hand reports,
has already been reported by L. da Ponte~\cite{daPonte}. \index{da Ponte}
 In his paper,
da Ponte attributes the mode of action of the magnetic force to direct
magnetic attraction.  This would be implausible with the relatively small
magnet used in this study; thus da Ponte was forced to propose an absurdly
large size for the  magnetic device.  His theory does not account for the
acoustic influence, which is known not to be present in magnetic action on
inanimate objects.  A second hypothesis would state that the acoustic
fields perturb the chemistry of the cyanide reactions, allowing the 
reagents to couple magnetically.   As with the first hypothesis, we are
skeptical, since again this phenomenon has no inorganic counterpart.  Thus,
we are forced to conclude, with Mesmer, that magnetism couples directly to 
the life force, which, as is well known, can be modulated by musical 
influences.

In conclusion, we note that the unusual mode of action of magnetism is only
one of many which have been reported in the literature.  It is likely
that there are further unusual magnetic phenomena to be discovered.  All
studies to date have involved male patients.  But it is likely that the 
prevailing belief that `women are all the same' would be overturned in a
similar investigation.




\Acknowledgements
I am grateful to Don Alfonso d'Alba for certain services essential to 
this investigation.



\begin{thebibliography}{99}

%%
%%  bibliographic items can be constructed using the LaTeX format in SPIRES:
%%    see    http://www.slac.stanford.edu/spires/hep/latex.html
%%  SPIRES will also supply the CITATION line information; please include it.
%%

\bibitem{Mesmer}
F. A. Mesmer, Proc. Wien. Acad. Sci. {\bf 13}, 1564, 1593 (1762).
%%CITATION = PWASA,13,1564;%%

\bibitem{diCenzo}
A. L. di Cenzo, Trans. Acad. Ducal.  Milan., {\bf 23}, 2647 (1771).
%%CITATION = TAADD,23,2647;%%

\bibitem{Muller}
For an exhaustive review of the Prussian literature, see A. D. M\"uller,
in Workshop on Action at a Distance, F. Eisenschmidt, ed. (Springer,
Berlin, 1774).

\bibitem{daPonte}

L. da Ponte, Trans. N. Y. Acad. Sci., {\bf 3}, 27 (1795).
%%CITATION = TNYAA,3,27;%%


\end{thebibliography}

 
\end{document}

